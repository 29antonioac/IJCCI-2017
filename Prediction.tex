\documentclass[a4paper,twoside]{article}

\usepackage{epsfig}
\usepackage{subfigure}
\usepackage{calc}
\usepackage{amssymb}
\usepackage{amstext}
\usepackage{amsmath}
\usepackage{amsthm}
\usepackage{multicol}
\usepackage{pslatex}
\usepackage{apalike}
\usepackage{SCITEPRESS}     % Please add other packages that you may need BEFORE the SCITEPRESS.sty package.

\subfigtopskip=0pt
\subfigcapskip=0pt
\subfigbottomskip=0pt

\begin{document}

\title{Title  \subtitle{Subtitle?} }

\author{\authorname{Antonio \'Alvarez-Caballero\sup{1}, J.J. Merelo\sup{1} and Third Author Name\sup{2}}
\affiliation{\sup{1}Department of Computer Architecture and Computer Technology, University of Granada, Spain}
\email{analca3@correo.ugr.es, jmerelo@ugr.es}
}

\keywords{Prediction, Classification, Strategy, Planification}

\abstract{The abstract should summarize the contents of the paper and should contain at least 70 and at most 200 words. The text must be set to 9-point font size.}

\onecolumn \maketitle \normalsize \vfill

\section{\uppercase{Introduction}}
\label{sec:introduction}

% Supervised learning is...

\noindent Supervised learning is a very useful approach to solve problems using
related data. In the field of AI applied to videogames research, its use is
conditional on the possibility of getting data from them.

% Introducing the game and its relation with supervised learning

Real Time Strategy or \emph{RTS} games are a very suitable kind of videogames
to use supervised learning with, because usually they have a very large set of
features which could be used to analyze the game deeply. In particular,
\emph{StarCraft} is a very well-known RTS game from the 90s. In this game both
players begin with a simple building and some workers, and the objective is
to create an army that can defeat the opponent's one. This is ussualy achieved
gathering resources from the map and building some important structures to get
the best units for your army. However, there are infinite strategies to follow,
and all of them are valid ones. Whatever strategy it is used, this process
generates a lot of data, which can be used to get hidden information from the
match.

% BWAPI: Accesing to data

Usually companies don't allow access matches' data to users. \emph{StarCraft}
is not an exception, but community have created an \emph{API} to access data
and manipulate the game itself: \emph{BWAPI}. With this tool users can create
artificial agent which play the game making competitions. Another common use
of this \emph{API} is gather data. The conclusions that can be extracted
from data could be very important, because could offer extra information
of matches in real time.

% Our proposal

In this paper we propose a set of \emph{StarCraft} features which can predict
the winner of a match precisely, with the help of a powerful classifier.
Furthermore, this features can be ranked to
get useful information of matches' flow. The framework we have chosen for this
work is the Apache ecosystem for data analytics, \emph{Spark} with the machine
learning library \emph{MLlib}. They provide tools able to deal with a big
amount of data, so we think it is a good decision due to the data dimension:
the analysis was done in an usual personal computer.

% What did we do?

In this work a complete Knowledge Discovery in Databases (\emph{KDD}) process
is done. The data were collected from \cite{DBLP:conf/flairs/RobertsonW14},
a set of six relational databases which contains a very big amount of data
from more than 4500 \emph{StarCraft} replays. A preprocessing with \emph{SQL}
was made to organise the data and extract our set of important features. At
last, the modelling was made using \emph{Spark} and \emph{MLlib} as we said,
allowing us to extract useful information as the winner in an early stage of
the matches and a ranking of useful features.

% What did we get?

We obtained some important conclussions: it is not needed to play the whole
match to know what player is the winner. With 10 minutes approximately, it is
enough to get precise predictions. Keeping in mind that the average duration
of a match is 48 minutes approximately, the time reduction is considerable. It
could be useful combined with metaheuristics to optimize agents for this
videogame faster.

These predictions are very precise. They have an accuracy rate above 90\%, so
we can conclude they are reliable. They could improve studies about this
videogame or bots, giving information about uncertainty in matches.

Another conclussion is that the set of features presented is very powerful.
Training a classifier is easy, but it doesn't help if data has no quality.
This set of data and features could be used in a lot of approaches.

\section{\uppercase{State of the art}}
\label{sec:state}

\noindent In the \emph{StarCraft} research a lot of approaches has been presented.
The most used approach is developing probabilistic graphical models to predict
the winner of a match. Some examples are in~\cite{DBLP:conf/cig/SynnaeveB11a}
and~\cite{DBLP:conf/aiide/StanescuHEGB13}, where important events in the match
are used to predict the outcome: when a very important building appears,
an important event succession for a race, the birth of the best unit of a race,
etc.

Another approach based in supervised learning is presented
in~\cite{DBLP:conf/cosecivi/Sanchez-Ruiz15}, but the environment is homogeneus
and controlled. It is possible that it doesn't show the diversity in
\emph{StarCraft} matches. A better dataset is presented
in~\cite{DBLP:conf/flairs/RobertsonW14}, which is very heterogeneus,
complete and granulated.

Another works look for plans and strategies based on predictions of the outcome
of matches, as we can see in~\cite{adaptativeStrategyPrediction} and
in~\cite{makingAndActing}.

Another approach is developing strategies using \emph{Genetic Programming},
creating plans automatically which can win. This kind of algorithms are very
time consuming, so whatever saved time would be appreciated. This approach
gives good results, as we can see in~\cite{DBLP:conf/evoW/Fernandez-AresG16}
and~\cite{DBLP:conf/cig/Garcia-SanchezT15}.

\section{\uppercase{Metodology}}
\label{sec:metodology}

\section{\uppercase{Results}}
\label{sec:results}

\section{\uppercase{Conclusions}}
\label{sec:conclusions}

\section*{\uppercase{Acknowledgements}}
\label{sec:acknowledgements}

\vfill
\bibliographystyle{apalike}
{\small
\bibliography{prediction}}

\vfill
\end{document}
